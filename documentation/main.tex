%%%%%%%%%%%%%%%%%%%%%%%%%%%%%%%%%%%%%%%%%
% Wenneker Assignment
% LaTeX Template
% Version 2.0 (12/1/2019)
%
% This template originates from:
% http://www.LaTeXTemplates.com
%
% Authors:
% Vel (vel@LaTeXTemplates.com)
% Frits Wenneker
%
% License:
% CC BY-NC-SA 3.0 (http://creativecommons.org/licenses/by-nc-sa/3.0/)
% 
%%%%%%%%%%%%%%%%%%%%%%%%%%%%%%%%%%%%%%%%%

%----------------------------------------------------------------------------------------
%	PACKAGES AND OTHER DOCUMENT CONFIGURATIONS
%----------------------------------------------------------------------------------------

\documentclass[11pt]{scrartcl} % Font size

\input{structure.tex} % Include the file specifying the document structure and custom commands

%----------------------------------------------------------------------------------------
%	TITLE SECTION
%----------------------------------------------------------------------------------------

\title{	
	\normalfont\normalsize
	\textsc{}\\ % Your university, school and/or department name(s)
	\vspace{25pt} % Whitespace
	\rule{\linewidth}{0.5pt}\\ % Thin top horizontal rule
	\vspace{20pt} % Whitespace
	{\huge Automating FEM solution database generation and neural network learning for unidimensional mechanical problems}\\ % The assignment title
	\vspace{12pt} % Whitespace
	\rule{\linewidth}{2pt}\\ % Thick bottom horizontal rule
	\vspace{12pt} % Whitespace
}

\author{\Large Mauricio Vanzulli, Jorge Perez Zerpa, Bruno Bazzano, Leopoldo Agorio} % Your name

\date{\normalsize\today} % Today's date (\today) or a custom date

\begin{document}

\maketitle % Print the title

%----------------------------------------------------------------------------------------
%	FIGURE EXAMPLE
%----------------------------------------------------------------------------------------

\section{Introduction}
The objective of this project is to develop a surrogate model for unidimensional mechanical problems that is faster than the finite element method (FEM). To achieve this, we trained a neural network using a dataset of FEM solutions to compression/extension mechanical problems. The ultimate goal is to apply this approach to develop state-of-the-art surrogate models for biological tissues. In this project, we used a FEM solver repeatedly to generate a large dataset of results, which was used to train the neural network. The neural network was designed to learn from the FEM solutions and solve similar mechanical problems more quickly. 

This documentation report will describe the methodology used to develop the neural network, the performance of the model compared to the FEM, and potential future directions for this project. By developing a faster surrogate model, this project has the potential to significantly improve the efficiency of solving unidimensional mechanical problems, with broader implications for the engineering design process.

\section{Image Interpretation}

\begin{figure}[h] % [h] forces the figure to be output where it is defined in the code (it suppresses floating)
	\centering
	\includegraphics[width=0.5\columnwidth]{swallow.jpg} % Example image
	\caption{European swallow.}
\end{figure}

%------------------------------------------------

\subsection{What is the airspeed velocity of an unladen swallow?}

While this question leaves out the crucial element of the geographic origin of the swallow, according to Jonathan Corum, an unladen European swallow maintains a cruising airspeed velocity of \textbf{11 metres per second}, or \textbf{24 miles an hour}. The velocity of the corresponding African swallows requires further research as kinematic data is severely lacking for these species.
\end{document}
